\documentclass[aspectratio=169]{beamer}

% Theme
\usetheme{Madrid}
\usecolortheme{default}

% Packages
\usepackage[french]{babel}
\usepackage[utf8]{inputenc}
\usepackage[T1]{fontenc}
\usepackage{graphicx}
\usepackage{listings}
\usepackage{xcolor}
\usepackage{booktabs}
\usepackage{amsmath}
\usepackage{tikz}

% Code listing style
\definecolor{codebg}{rgb}{0.95, 0.95, 0.95}
\lstset{
    backgroundcolor=\color{codebg},
    frame=single,
    framerule=0.5pt,
    basicstyle=\ttfamily\small,
    breaklines=true,
    numbers=none,
    tabsize=2
}

% Title page information
\title[LogiTrack]{LogiTrack}
\subtitle{Système de Gestion des Livraisons avec Oracle Database}
\author[Said Jadli, Yasser Nadi, Adnane Qassiri]{
    Said Jadli \\
    Yasser Nadi \\
    Adnane Qassiri
}
\institute[ENSA]{École Nationale des Sciences Appliquées}
\date{2025/2026}

% Logo
\titlegraphic{\includegraphics[width=2cm]{images/logo-ensa.png}}

\begin{document}

% Title frame
\begin{frame}
    \titlepage
\end{frame}

% Table of contents
\begin{frame}{Plan de la présentation}
    \tableofcontents
\end{frame}

% Section 1: Introduction
\section{Introduction}

\begin{frame}{Contexte et Objectifs}
    \begin{block}{Contexte}
        \begin{itemize}
            \item Défis de la logistique moderne : traçabilité, optimisation, satisfaction client
            \item Besoin d'un système complet de gestion des livraisons
            \item Gestion multi-entrepôts avec différents acteurs
        \end{itemize}
    \end{block}
    
    \begin{block}{Objectifs}
        \begin{itemize}
            \item Développer un système de gestion des livraisons complet
            \item Exploiter les fonctionnalités avancées d'Oracle Database
            \item Créer une interface web moderne et intuitive
            \item Garantir la traçabilité et la cohérence des données
        \end{itemize}
    \end{block}
\end{frame}

\begin{frame}{Vue d'ensemble du système}
    \begin{columns}
        \column{0.5\textwidth}
        \textbf{LogiTrack} est un système de gestion logistique qui permet :
        \begin{itemize}
            \item Gestion des colis
            \item Gestion des livraisons
            \item Gestion des entrepôts
            \item Suivi en temps réel
            \item Tableaux de bord KPI
        \end{itemize}
        
        \column{0.5\textwidth}
        \begin{block}{Trois rôles principaux}
            \begin{itemize}
                \item \textbf{Admin} : Gestion globale
                \item \textbf{Gestionnaire} : Gestion d'entrepôt
                \item \textbf{Livreur} : Livraisons
            \end{itemize}
        \end{block}
    \end{columns}
\end{frame}

% Section 2: Architecture
\section{Architecture}

\begin{frame}{Architecture 3-Tiers}
    \begin{center}
        \begin{tikzpicture}[node distance=1.5cm]
            \node[draw, rectangle, fill=blue!20, minimum width=3.5cm, minimum height=1.2cm, text centered] (frontend) {\textbf{Frontend}\\\small React};
            \node[draw, rectangle, fill=green!20, minimum width=3.5cm, minimum height=1.2cm, text centered, below of=frontend] (backend) {\textbf{Backend}\\\small Node.js/Express};
            \node[draw, rectangle, fill=orange!20, minimum width=3.5cm, minimum height=1.2cm, text centered, below of=backend] (database) {\textbf{Base de données}\\\small Oracle};
            
            \draw[->, thick, blue] (frontend) -- node[right] {\small API REST} (backend);
            \draw[->, thick, green] (backend) -- node[right] {\small PL/SQL} (database);
        \end{tikzpicture}
    \end{center}
    
    \vspace{0.5cm}
    \begin{block}{Avantages}
        \begin{itemize}
            \item Séparation claire des responsabilités
            \item Scalabilité et maintenabilité
            \item Sécurité renforcée
            \item Réutilisabilité des composants
        \end{itemize}
    \end{block}
\end{frame}

\begin{frame}{Stack Technologique}
    \begin{columns}
        \column{0.33\textwidth}
        \begin{block}{Frontend}
            \begin{itemize}
                \item React 18.2
                \item Tailwind CSS
                \item React Router
                \item Axios
            \end{itemize}
        \end{block}
        
        \column{0.33\textwidth}
        \begin{block}{Backend}
            \begin{itemize}
                \item Node.js
                \item Express.js
                \item OracleDB
                \item Express-session
            \end{itemize}
        \end{block}
        
        \column{0.33\textwidth}
        \begin{block}{Base de données}
            \begin{itemize}
                \item Oracle Database
                \item PL/SQL
                \item Triggers
                \item Packages
            \end{itemize}
        \end{block}
    \end{columns}
\end{frame}

% Section 3: Base de données
\section{Base de données}

\begin{frame}{Modèle de données}
    \begin{block}{8 Tables principales}
        \begin{columns}
            \column{0.5\textwidth}
            \begin{itemize}
                \item \textbf{UTILISATEURS}
                \item \textbf{ENTREPOTS}
                \item \textbf{CLIENTS}
                \item \textbf{VEHICULES}
            \end{itemize}
            
            \column{0.5\textwidth}
            \begin{itemize}
                \item \textbf{LIVRAISONS}
                \item \textbf{COLIS}
                \item \textbf{HISTORIQUE\_STATUT\_COLIS}
                \item \textbf{HISTORIQUE\_STATUT\_LIVRAISONS}
            \end{itemize}
        \end{columns}
    \end{block}
    
    \begin{block}{Relations principales}
        \begin{itemize}
            \item Un gestionnaire par entrepôt (1-1)
            \item Un livreur peut avoir plusieurs livraisons (1-N)
            \item Un colis appartient à un client et une livraison
            \item Traçabilité complète via tables d'historique
        \end{itemize}
    \end{block}
\end{frame}

\begin{frame}{Fonctionnalités Oracle}
    \begin{columns}
        \column{0.5\textwidth}
        \begin{block}{Séquences}
            \begin{itemize}
                \item 8 séquences pour génération automatique d'IDs
                \item Utilisées par triggers BEFORE INSERT
            \end{itemize}
        \end{block}
        
        \begin{block}{Triggers}
            \begin{itemize}
                \item Calcul automatique du prix
                \item Assignation automatique à livraison
                \item Gestion des statuts
                \item Synchronisation gestionnaire-entrepôt
            \end{itemize}
        \end{block}
        
        \column{0.5\textwidth}
        \begin{block}{Package PL/SQL}
            \begin{itemize}
                \item \texttt{pkg\_logitrack}
                \item 15+ procédures métier
                \item Contrôle des rôles
                \item Gestion des erreurs
            \end{itemize}
        \end{block}
        
        \begin{block}{Vues}
            \begin{itemize}
                \item v\_livraisons\_details
                \item v\_colis\_details
                \item v\_vehicules\_entrepots
                \item v\_kpi\_dashboard
            \end{itemize}
        \end{block}
    \end{columns}
\end{frame}

\begin{frame}{Exemple de Trigger}
    \begin{block}{Calcul automatique du prix et assignation}
        \begin{lstlisting}[language=SQL, basicstyle=\tiny]
CREATE OR REPLACE TRIGGER trg_colis_assign_price
BEFORE INSERT OR UPDATE ON colis
FOR EACH ROW
DECLARE
  v_base NUMBER := 20;
  v_dest_entrepot NUMBER;
BEGIN
  -- Calcul du prix
  IF :NEW.type_colis = 'FRAGILE' THEN
    v_base := 30;
  END IF;
  :NEW.prix := ROUND(:NEW.poids * v_base, 2);
  
  -- Assignation automatique à une livraison
  -- Trouve ou crée une livraison selon la destination
END;
        \end{lstlisting}
    \end{block}
    
    \begin{itemize}
        \item Prix calculé : 20 MAD/kg (STANDARD) ou 30 MAD/kg (FRAGILE)
        \item Assignation automatique à une livraison selon la destination
        \item Création automatique de livraison si nécessaire
    \end{itemize}
\end{frame}

% Section 4: Fonctionnalités
\section{Fonctionnalités}

\begin{frame}{Rôle : Administrateur}
    \begin{block}{Tableau de bord}
        \begin{itemize}
            \item 8 KPI en temps réel
            \begin{itemize}
                \item Total Clients, Colis, Livraisons
                \item Livreurs actifs, Entrepôts
                \item Admins, Gestionnaires
                \item Chiffre d'affaires
            \end{itemize}
        \end{itemize}
    \end{block}
    
    \begin{block}{Gestion}
        \begin{itemize}
            \item Utilisateurs (création, modification, activation)
            \item Clients
            \item Entrepôts et gestionnaires
            \item Véhicules
        \end{itemize}
    \end{block}
\end{frame}

\begin{frame}{Rôle : Gestionnaire}
    \begin{columns}
        \column{0.5\textwidth}
        \begin{block}{Colis Envoyés}
            \begin{itemize}
                \item Visualisation des colis envoyés
                \item Filtrage par statut
                \item Modification de statut
                \item Annulation (si non envoyé)
            \end{itemize}
        \end{block}
        
        \begin{block}{Colis Reçus}
            \begin{itemize}
                \item Visualisation des colis reçus
                \item Marquage comme récupéré (via CIN)
                \item Filtrage (Reçus, Récupérés, Tous)
            \end{itemize}
        \end{block}
        
        \column{0.5\textwidth}
        \begin{block}{Autres fonctionnalités}
            \begin{itemize}
                \item Enregistrement de nouveaux colis
                \item Gestion des clients
                \item Gestion des véhicules de l'entrepôt
                \item Statistiques de l'entrepôt
            \end{itemize}
        \end{block}
    \end{columns}
\end{frame}

\begin{frame}{Rôle : Livreur}
    \begin{block}{Livraisons disponibles}
        \begin{itemize}
            \item Visualisation des livraisons depuis son entrepôt
            \item Filtrage : uniquement celles avec au moins 1 colis
            \item Prise en charge avec sélection du véhicule
        \end{itemize}
    \end{block}
    
    \begin{block}{Mes livraisons}
        \begin{itemize}
            \item Visualisation des livraisons en cours
            \item Détails : source, destination, nombre de colis
            \item Marquage comme livrée
            \item Statistiques personnelles
        \end{itemize}
    \end{block}
    
    \begin{block}{Automatismes}
        \begin{itemize}
            \item Mise à jour automatique du statut véhicule
            \item Mise à jour automatique du statut des colis
            \item Création automatique d'une nouvelle livraison
        \end{itemize}
    \end{block}
\end{frame}

% Section 5: Sécurité et Performance
\section{Sécurité et Performance}

\begin{frame}{Sécurité}
    \begin{columns}
        \column{0.5\textwidth}
        \begin{block}{Authentification}
            \begin{itemize}
                \item Sessions serveur (express-session)
                \item Cookies httpOnly
                \item Vérification des rôles
            \end{itemize}
        \end{block}
        
        \begin{block}{Autorisation}
            \begin{itemize}
                \item Middleware de contrôle d'accès
                \item Vérification dans les procédures PL/SQL
                \item Routes protégées frontend
            \end{itemize}
        \end{block}
        
        \column{0.5\textwidth}
        \begin{block}{Intégrité des données}
            \begin{itemize}
                \item Contraintes CHECK
                \item Contraintes UNIQUE
                \item Clés étrangères
                \item Validation dans triggers
            \end{itemize}
        \end{block}
        
        \begin{block}{Traçabilité}
            \begin{itemize}
                \item Historiques complets
                \item Identification des acteurs
                \item Timestamps automatiques
            \end{itemize}
        \end{block}
    \end{columns}
\end{frame}

\begin{frame}{Optimisations}
    \begin{block}{Base de données}
        \begin{itemize}
            \item Pool de connexions Oracle (min: 1, max: 5)
            \item Vues pré-calculées pour KPI
            \item Index uniques sur colonnes critiques
            \item Requêtes optimisées avec JOINs
        \end{itemize}
    \end{block}
    
    \begin{block}{Backend}
        \begin{itemize}
            \item Logique métier dans procédures stockées
            \item Curseurs Oracle (SYS\_REFCURSOR)
            \item Gestion asynchrone (async/await)
        \end{itemize}
    \end{block}
    
    \begin{block}{Frontend}
        \begin{itemize}
            \item Composants React optimisés
            \item Chargement conditionnel
            \item Requêtes ciblées
        \end{itemize}
    \end{block}
\end{frame}

% Section 6: Démonstration
\section{Démonstration}

\begin{frame}{Flux de travail : Enregistrement d'un colis}
    \begin{enumerate}
        \item \textbf{Gestionnaire} enregistre un nouveau colis
        \begin{itemize}
            \item Saisie : client, poids, type, destinataire, destination
        \end{itemize}
        \item \textbf{Trigger automatique} :
        \begin{itemize}
            \item Calcule le prix (poids × tarif)
            \item Trouve ou crée une livraison vers la destination
            \item Assigne le colis à la livraison
        \end{itemize}
        \item \textbf{Colis} apparaît dans "Colis Envoyés" avec statut ENREGISTRE
    \end{enumerate}
\end{frame}

\begin{frame}{Flux de travail : Livraison}
    \begin{enumerate}
        \item \textbf{Livreur} consulte les livraisons disponibles
        \item \textbf{Livreur} prend en charge une livraison
        \begin{itemize}
            \item Sélectionne un véhicule disponible
        \end{itemize}
        \item \textbf{Trigger automatique} :
        \begin{itemize}
            \item Statut livraison : CREEE → EN\_COURS
            \item Statut véhicule : DISPONIBLE → EN\_UTILISATION
            \item Statut colis : ENREGISTRE → EN\_COURS
            \item Crée les historiques
        \end{itemize}
        \item \textbf{Livreur} livre la livraison
        \item \textbf{Trigger automatique} :
        \begin{itemize}
            \item Statut livraison : EN\_COURS → LIVREE
            \item Statut véhicule : EN\_UTILISATION → DISPONIBLE
            \item Statut colis : EN\_COURS → LIVRE
            \item Crée une nouvelle livraison CREEE pour la même route
        \end{itemize}
    \end{enumerate}
\end{frame}

\begin{frame}{Flux de travail : Récupération}
    \begin{enumerate}
        \item \textbf{Colis} arrive à destination (statut : LIVRE)
        \item \textbf{Colis} apparaît dans "Colis Reçus" du gestionnaire de destination
        \item \textbf{Destinataire} se présente avec son CIN
        \item \textbf{Gestionnaire} marque le colis comme récupéré
        \begin{itemize}
            \item Vérification du CIN
        \end{itemize}
        \item \textbf{Statut} : LIVRE → RECUPEREE
        \item \textbf{Colis} comptabilisé dans le chiffre d'affaires
    \end{enumerate}
\end{frame}

% Section 7: Résultats
\section{Résultats}

\begin{frame}{Objectifs atteints}
    \begin{block}{✓ Conception}
        \begin{itemize}
            \item Base de données relationnelle normalisée (8 tables)
            \item Modèle de données cohérent et complet
        \end{itemize}
    \end{block}
    
    \begin{block}{✓ Implémentation Oracle}
        \begin{itemize}
            \item Logique métier dans packages PL/SQL
            \item Automatisation via triggers
            \item Vues pour simplifier les requêtes
        \end{itemize}
    \end{block}
    
    \begin{block}{✓ Application web}
        \begin{itemize}
            \item Interface moderne et intuitive (React)
            \item Gestion des rôles et permissions
            \item Tableaux de bord KPI
        \end{itemize}
    \end{block}
    
    \begin{block}{✓ Traçabilité}
        \begin{itemize}
            \item Historiques complets des changements
            \item Identification des acteurs
        \end{itemize}
    \end{block}
\end{frame}

\begin{frame}{Points forts}
    \begin{columns}
        \column{0.5\textwidth}
        \begin{block}{Cohérence}
            \begin{itemize}
                \item Logique métier centralisée
                \item Intégrité garantie par Oracle
                \item Automatisation complète
            \end{itemize}
        \end{block}
        
        \begin{block}{Sécurité}
            \begin{itemize}
                \item Contrôle d'accès multi-niveaux
                \item Validation des données
                \item Traçabilité complète
            \end{itemize}
        \end{block}
        
        \column{0.5\textwidth}
        \begin{block}{Performance}
            \begin{itemize}
                \item Pool de connexions
                \item Vues optimisées
                \item Requêtes efficaces
            \end{itemize}
        \end{block}
        
        \begin{block}{Maintenabilité}
            \begin{itemize}
                \item Code structuré
                \item Documentation claire
                \item Architecture modulaire
            \end{itemize}
        \end{block}
    \end{columns}
\end{frame}

% Section 8: Perspectives
\section{Perspectives}

\begin{frame}{Évolutions possibles}
    \begin{block}{Fonctionnalités}
        \begin{itemize}
            \item Notifications email/SMS
            \item Suivi GPS en temps réel
            \item Module de facturation
            \item Application mobile pour livreurs
            \item Portail client
        \end{itemize}
    \end{block}
    
    \begin{block}{Techniques}
        \begin{itemize}
            \item Hashage des mots de passe (bcrypt)
            \item Mise en cache Redis
            \item Tests automatisés
            \item CI/CD pipeline
            \item Monitoring avancé
        \end{itemize}
    \end{block}
    
    \begin{block}{Métier}
        \begin{itemize}
            \item Optimisation des routes
            \item Gestion des stocks
            \item Intégration transporteurs externes
            \item Module qualité
        \end{itemize}
    \end{block}
\end{frame}

% Section 9: Conclusion
\section{Conclusion}

\begin{frame}{Conclusion}
    \begin{block}{Réalisations}
        \begin{itemize}
            \item Système complet de gestion des livraisons
            \item Exploitation avancée d'Oracle Database
            \item Interface web moderne et intuitive
            \item Automatisation des processus métier
            \item Traçabilité complète
        \end{itemize}
    \end{block}
    
    \begin{block}{Apports pédagogiques}
        \begin{itemize}
            \item Modélisation relationnelle
            \item Programmation PL/SQL avancée
            \item Intégration base de données / application
            \item Architecture 3-tiers
        \end{itemize}
    \end{block}
    
    \vspace{0.5cm}
    \begin{center}
        \Large \textbf{Merci pour votre attention !}
    \end{center}
\end{frame}

\begin{frame}{Questions ?}
    \begin{center}
        \Huge \textbf{Questions ?}
        
        \vspace{1cm}
        
        \Large Contact : \\
        \vspace{0.5cm}
        Said Jadli, Yasser Nadi, Adnane Qassiri
    \end{center}
\end{frame}

\end{document}

